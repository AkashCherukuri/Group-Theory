\section{Subgroups}

\subsection{Definitions and Basics}
Although we have defined subgroups in section $1$, we repeat the definition here.
\begin{definition}
Let $G$ be a group. A subset $H$ of $G$ is a subgroup of $G$ if $H$ is nonempty and it is closed under products and inverses. That is, $x,y\in H$ implies $x^{-1}\in H$ and $xy\in H$. If $H$ is a subgroup of $G$, we write $H\leq G$.
\end{definition}

If $H\leq G$ and $H\neq G$, we write $H<G$.

\begin{example}
    $\mathbb{Z}\leq\mathbb{Q}$ and $\mathbb{Q}\leq\mathbb{R}$ under the operation of addition.
    
    If $G=D_{2n}$, $H=\{1,r,r^2,\ldots,r^{n-1}\}$ is a subgroup of $G$.
\end{example}

Note that the relation $\leq$ is transitive. That is, if $K\leq H$ and $H\leq G$, then $K\leq G$.

\begin{theorem}[Subgroup Criterion]
A subset $H$ of a group $G$ is a subgroup if and only if
\begin{enumerate}
    \item $H\neq\emptyset$.
    \item for all $x,y\in H$, $xy^{-1}\in H$.
\end{enumerate}
Further, if $H$ is finite, then it suffices to check that $H$ is nonempty and is closed under multiplication.
\end{theorem}
\begin{proof}
    If $H\leq G$, the two given statements clearly hold as $H$ contains the identity of $G$ and is closed under inverses and multiplication.
    
    To prove the converse, let $x$ be any element of $H$ (which exists as $H\neq\emptyset$). We have $xx^{-1}\in H\implies 1\in H$. As $H$ contains $1$, for any element $h$ of $H$, $H$ contains $1h^{-1}=h^{-1}$, that is, it is closed under inverses. For any $x$ and $y$ in $H$, as $y^{-1}\in H$, we have that $x(y^{-1})^{-1}=xy\in H$, that is, $H$ is closed under multiplication.
    
    \vspace{1mm}
    To prove the second part, we see that $x,x^2,x^3,\ldots\in H$ for any $x\in H$. Using \ref{finGrpFinOrd}, we see that $x$ is of finite order $n$. Then $x^{-1}=x^{n-1}\in H$ so $H$ is closed under inverses.
\end{proof}

\begin{exercise}
    Let $G$ be a group and $H,K$ be subgroups of $G$. Show that $H\cup K$ is a subgroup if and only if $H\subseteq K$ or $K\subseteq H$.
\end{exercise}

\begin{exercise}
    Let $G$ be a group and $H,K$ be subgroups of $G$. Show that $H\cap K$ is also a subgroup of $G$.
\end{exercise}

\begin{exercise}
\label{subgrIntersubgr}
    Let $G$ be a group. Prove that the intersection of an arbitrary nonempty collection of subgroups of $G$ is again a subgroup of $G$.
\end{exercise}

\begin{exercise}
    Let $G$ be a group of order $n>2$. Show that $G$ cannot have a subgroup $H$ of order $n-1$.
\end{exercise}

\begin{exercise}
    Let $G$ be a group. Let $H=\{g\in G\mid |g|<\infty\}$. Show that $H\leq G$ if $G$ is abelian. In this case, $H$ is called the \textit{torsion subgroup} of $G$. Give an example where $G$ is non-abelian and $H$ is not a subgroup of $G$.
\end{exercise}

\begin{exercise}
    Let $H$ be a subgroup of $\mathbb{Q}$ under addition with the property that $\frac 1x\in H$ for every nonzero $x\in H$. Show that $H=\{0\}$ or $\mathbb{Q}$.
\end{exercise}

\subsection{Centralizers, Normalizers, Stabilizers and Kernels}

We now introduce some important subgroups.

\begin{definition}
    Let $G$ be a group and $A$ be any nonempty subset of $A$. Define
    $$C_G(A)=\{g\in G\mid gag^{-1}=a\text{ for all }a\in A\}.$$
    This subset is called the \textit{centralizer} of $A$ in $G$.
\end{definition}

Since $gag^{-1}=g$ if and only if $ga=ag$, $C_G(A)$ is the set of all elements that commute with every element of $A$.

Now observe that $C_G(A)$ is a subgroup of $G$ as first of all, $1\in C_G(A)$ so $C_G(A)\neq\emptyset$, and second of all, if $x,y\in C_G(A)$, we have $xax^{-1}=a$ and $yay^{-1}=a$, that is, $y^{-1}ay=a$ for all $a\in A$. We then have $a=xax^{-1}=x(y^{-1}ay)x^{-1}=(xy^{-1})a(xy^{-1})^{-1}$ so $xy^{-1}\in C_G(A)$. Thus, $C_G(A)\leq G$.

\begin{definition}
    Let $G$ be a group. Define
    $$Z(G)=\{g\in G\mid gx=xg\text{ for all }x\in G\}.$$
    This subset is called the \textit{center} of $G$.
\end{definition}

$Z(G)$ is the set of all elements that commute with every element of $G$.

As $Z(G)=C_G(G)$, we have $Z(G)\leq G$.

\begin{definition}
    Let $G$ be a group and $A$ be a subset of $G$. Define $gAg^{-1}=\{gag^{-1}\mid a\in A\}$. Define
    $$N_G(A)=\{g\in G\mid gAg^{-1}=A\}.$$
    This set is called the \textit{normalizer} of $A$ in $G$.
\end{definition}

The proof that $N_G(A)\leq G$ is similar to that we used to prove that $C_G(A)\leq G$.

Note that $C_G(A)\leq N_G(A)$.

\vspace{2mm}
If $G$ is an abelian group, $Z(G)=G$. Further, for any subset $A$ of $G$, $N_G(A)=C_G(A)=G$ as $gag^{-1}=gg^{-1}a=a$ for all $a\in A, g\in G$.

\begin{exercise}
    Show that the center of $D_8$ is $\{1,r^2\}$.
\end{exercise}

The fact that centralizers and normalizers are subgroups is in fact a special case of a results in group actions. 

We now introduce stablizers and kernels of group actions.

\begin{definition}
    Let $G$ be a group that acts on a set $S$. Let $s\in S$ be some fixed elements. Define
    $$G_s=\{g\in G\mid g\cdot s=s\}$$
\end{definition}

We shall now show that $G_s\leq G$. First of all, $1\in G_s$ by the definition of a group action. If $x,y\in G_s$, we have
\begin{align*}
    s &= 1\cdot s \\
      &= (x^{-1}x)\cdot s \\
      &= x^{-1}\cdot (x\cdot s) \\
      &= x^{-1}\cdot s
\end{align*}
so $x^{-1}\in G_s$ and
\begin{align*}
    (xy)\cdot s &= x\cdot(y\cdot s) \\
                &= x\cdot s \\
                &= s
\end{align*}
We see that $G_s$ is nonempty and is closed under inverses and multiplication. It is thus a subgroup of $G$.

Recall the definition of a \textit{kernel} of an action, \ref{defKerGrpAc}. Using \ref{subgrIntersubgr} and the fact that $G_s\leq G$ for all $s\in S$ yields the result that the kernel of any group action is a subgroup of the group.

\vspace{3mm}
We now see that $C_G(A)$ is merely the kernel of the group action of $G$ acting on $A$ as $g\cdot a= gag^{-1}$ (so it is a subgroup of $G$) and $N_G(A)$ is the stabilizer of the group action of $G$ acting on $\mathcal{P}(A)$ (the power set of $A$) as $g\cdot A=gAg^{-1}$ (so it is a subgroup of $G$).

\begin{exercise}
    Prove that $C_G(Z(G))=N_G(Z(G))=G$.
\end{exercise}

\begin{exercise}
    Prove that $H\leq N_G(H)$ for a subgroup $H$ of a group $G$.
\end{exercise}

\begin{exercise}
    For any subgroup $H$ of group $G$ and subset $A$ of $G$, define $N_H(A)=\{h\in H\mid hAh^{-1}=A\}$. Prove that $N_H(A)=N_G(A)\cap H$ and deduce that $N_H(A)\leq N_G(A)$.
\end{exercise}

\begin{exercise}
    Let $F$ be a field and the Heisenberg group $H(F)$ be defined as in \ref{defOfHeisenbergGrp}. Determine $Z(H(F))$ and prove that $Z(H(F))\cong (F,+)$.
\end{exercise}

\subsection{Cyclic Groups and Cyclic Subgroups}

\begin{definition}
    A group $H$ is \textit{cyclic} if there is some element $x\in H$ such that $H=\{x^n\mid n\in\mathbb{Z}\}$.
\end{definition}

In this case we write $H=\langle x\rangle$ and say that $H$ is \textit{generated} by $x$ and $x$ is a generator of $H$. The generator of a cyclic group need not be unique (as if $x$ is a generator, so is $-x$).

Note that any cyclic group is abelian.

\begin{example}
    The group $(\mathbb{Z},+)$ is generated by $1$ (here $1$ is the integer $1$ and not the identity).
\end{example}

\begin{theorem}
\label{orderOfCycGrpisOrderOfGen}
    Let $H=\langle x\rangle$. Then $|H|=|x|$ (where if one side of the inequality is infinite, so is the other).
\end{theorem}

\begin{proof}
    This proof is trivial and is left as an exercise to the reader.
\end{proof}

It is observed that there is a great deal of similarity between $H=\langle x\rangle$, where $|x|=n$, and $\mathbb{Z}/n\mathbb{Z}$. Both of them appear to have very similar structure. It turns out that these two groups are isomorphic, which we shall prove shortly. First, let us prove the following.

\begin{theorem}
\label{orderGCD}
    Let $G$ be an arbitrary group, $x\in G$, and let $m,n\in\mathbb{Z}$. If $x^n=1$ and $x^m=1$, then $x^d=1$, where $d=(m,n)$. In particular, if $x^m=1$ for some $m\in\mathbb{Z}$, then $|x|\mid m$.
\end{theorem}

\begin{proof}
    By the Euclidean algorithm, there exist integers $r$ and $s$ such that $d=mr+ns$. We have
    $$x^d=x^{mr+ns}=(x^m)^r(x^n)^s=1.$$
    This proves our first claim.
    
    Next, let $n=|x|$ and $x^m=1$. We have $x^d$=1, where $d=(|x|,m)$. Note that $0<d\leq |x|$ and $|x|$ is the smallest positive integer $k$ such that $x^k=1$. This implies that $d=|x|$ and $|x|=(|x|,m)$. Thus, $|x|\mid m$. 
\end{proof}

\begin{theorem}
    Any two cyclic groups of the same order are isomorphic. More specifically,
    \begin{enumerate}
        \item if $n\in\mathbb{Z}^+$ and $H=\langle x\rangle$ and $K=\langle y\rangle$ are both of order $n$, $H\cong K$.
        \item if $\langle x\rangle$ is an infinite cyclic group, $(\mathbb{Z},+)\cong\langle x\rangle$.
    \end{enumerate}
\end{theorem}
\begin{proof}
Let $\langle x\rangle$ and $\langle y\rangle$ be two cyclic groups of finite order $n$. Let $\varphi:\langle x\rangle\to\langle y\rangle$ be defined by $\phi(x^k)=y^k$. Let us first prove that $\varphi$ is well defined, that is, if $x^a=x^b$, then $\varphi(x^a)=\varphi(x^b)$. If $x^a=x^b$, $x^{b-a}=1$ and \ref{orderGCD} implies that $n\mid b-a$. Let $b=a+tn$ so
$\varphi(x^b)=\varphi(x^{a+tn})=y^{a+tn}=(y^n)^ty^a=y^a=\varphi(x^a)$. Thus $\varphi$ is well-defined. $\varphi$ is a homomorphism as $\varphi(x^a)\varphi(x^b)=y^ay^b=y^{a+b}=\varphi(x^ax^b)$. $\varphi$ is injective as any element $y^a$ of $\langle y\rangle$ is the image of $x^a$. As $\varphi$ is a surjection between two sets of equal finite order, it is a bijection and $\varphi$ is an isomorphism.

\vspace{2mm}
Let $\langle x\rangle$ be an infinite cyclic group. Consider the map $\varphi:(\mathbb{Z},+)\to\langle x\rangle$ given by $\varphi(k)=x^k$ for $k\in\mathbb{Z}$. This function is a homomorphism as $\varphi(a)\varphi(b)=x^ax^b=x^{a+b}=\varphi(a+b)$. Since $x^a\neq x^b$ for $a\neq b$, $\varphi$ is an injection. As any element $x^a\in\langle x\rangle$ is the image of $a\in\mathbb{Z}$, $\varphi$ is a surjection. Thus $\varphi$ is a bijection and an isomorphism.
\end{proof}

For each $n\in\mathbb{Z}^+$, let $\mathbb{Z}_n$ be the cyclic group of order $n$. $\mathbb{Z}_n\cong\mathbb{Z}/n\mathbb{Z}$.

\begin{theorem}
\label{orderIsOrderByGCD}
    Let $G$ be a group, $x\in G$ and $a\in\mathbb{Z}-\{0\}$.
    \begin{enumerate}
        \item If $|x|=\infty$, $|x^a|=\infty$.
        \item If $|x|=n<\infty$, $|x^a|=\frac{n}{(n,a)}$.
    \end{enumerate}
\end{theorem}

\begin{proof}
    \phantom{owo}
    \begin{enumerate}
        \item On the contrary, let $|x^a|=k<\infty$. Then $(x^a)^k=x^{ak}=1$. Also $x^{-ak}=1$. Since one of $ak$ and $-ak$ must be positive, some positive power of $x$ is $1$, which contradicts the fact that $|x|=\infty$. Thus, $|x^a|=\infty$.
        \item Let $y=x^a, d=(n,a), a=bd$ and $n=cd$ for some $b,c\in\mathbb{Z}$. We must show that $|y|=c$. We have $y^c=(x^a)^c=(x^{bd})^c=(x^{cd})^b=(x^n)^b=1$. \ref{orderGCD} implies that $|y|\mid c$. We also have $x^{a|y|}=1$ which implies that $|x|\mid a|y|$. This gives $cd\mid bd|y|$, that is, $c\mid b|y|$. However, since $(b,c)=1$, we have $c\mid |y|$. As $|y|\mid c$ and $c\mid |y|$, $|y|=c$.
    \end{enumerate}
\end{proof}

\begin{corollary}
\label{ifDivThenOrderIsDiv}
    A corollary of the second part of the above theorem is that if $a\mid n$, $|x^a|=\frac na$.
\end{corollary}

\begin{exercise}
    Assume $|x|=n<\infty$. Then $H=\langle x^a\rangle$ if and only if $(a,n)=1$.
\end{exercise}
\begin{proof}
    We have that $x^a$ generates a group of order $|x^a|$. This subgroup equals $H$ if and only if $|x^a|=|x|$, that is, $\frac{n}{(a,n)}=n$. This is equivalent to $(a,n)=1$.
\end{proof}

This implies that the total number of generators of a cyclic group of order $n$ is $\varphi(n)$, where $\varphi$ is Euler's totient function.

\begin{theorem}
    Let $H=\langle x\rangle$ be a cyclic group.
    \begin{enumerate}
        \item Every subgroup of $H$ is cyclic. More precisely, if $K\leq H$, either $K=\{1\}$ or $K=\langle x^d\rangle$, where $d$ is the smallest positive integer such that $x^d\in K$.
        \item If $|H|=\infty$, then for distinct nonnegative integers $a,b$, $\langle x^a\rangle\neq\langle x^b\rangle$. Also, $\langle x^m\rangle=\langle x^{|m|}\rangle$ so the nontrivial subgroups of $H$ are in bijection with $\mathbb{N}$.
        \item If $|H|=n<\infty$, then for each positive integer $a$ dividing $n$, there is a unique subgroup of $H$ of order $a$, namely $\langle x^{n/a}\rangle$. Furthermore, for every integer $m$, $\langle x^m\rangle=\langle x^{(n,m)}\rangle$. (So the subgroups of $H$ are in bijection with the positive integers of $n$)
    \end{enumerate}
\end{theorem}

\begin{proof}
    \phantom{beegyoshi}
    \begin{enumerate}
        \item Let $d$ be the smallest positive integer such that $x^d\in K$. As $K$ is a group, $x^kd\in K$ for any $k\in\mathbb{Z}$. Let $x^a\in K$ for some $a\in\mathbb{Z}$. Write $a=qd+r$ where $q,r\in\mathbb{Z}$ and $0\leq r<d$. Then $x^r=x^{a}x^{-qd}\in K$ as $K$ is a group. However, by the minimality of $d$ and the fact that $0\leq r<d$, we get $r=0$. As $d$ divides any $a$ such that $x^a\in K$ and $\langle x^d\rangle\leq K$, we have $K=\langle x^d\rangle$.
        \item This proof is similar to that of the third part so we leave it as an exercise to the reader.
        \item Use \ref{ifDivThenOrderIsDiv} to get that $|x^{n/a}|=a$, which gives that $\langle x^{n/a}\rangle$ is of order $a$. We must now prove that this is the unique subgroup of order $a$. Let $b\in\mathbb{Z}$ such that $\langle x^b\rangle$ is of order $a$. We have that the order of $\langle x^b\rangle$ is equal to $|x^b|$ from \ref{orderOfCycGrpisOrderOfGen}. Using \ref{orderIsOrderByGCD} gives $a=\frac{n}{(n,b)}$ so $\frac na=(n,b)$. In particular, $\frac na\mid b$. This implies that $\langle x^b\rangle\leq\langle x^\frac{n}{a}\rangle$. However, since they are of equal finite order, $\langle x^b\rangle=\langle x^\frac{n}{a}\rangle$ and $\langle x^\frac{n}{a}\rangle$ is the unique subgroup of order $a$. 
    \end{enumerate}
\end{proof}

\begin{exercise}
    Let $p$ be a prime and $n\in\mathbb{Z}^+$. Show that if $x$ is an element of a group $G$ such that $x^{p^n}=1$, then $|x|=p^m$ for some $m\leq n$.
\end{exercise}

\begin{exercise}
    Prove that $\mathbb{Z}_2\times\mathbb{Z}_2$, $\mathbb{Z}_2\times\mathbb{Z}$ and $\mathbb{Z}\times\mathbb{Z}$ are not cyclic.
\end{exercise}

\begin{exercise}
    Let $G$ be a group and $x\in G$. Prove that $g\in N_G(\langle x\rangle)$ if and only if $gxg^{-1}=x^a$ for some $a\in\mathbb{Z}$.
\end{exercise}

\begin{exercise}
    Show that $(\mathbb{Z}/2^n\mathbb{Z})^\times$ is not cyclic for any $n\geq 3$.
\end{exercise}


\clearpage