\documentclass{article}

\usepackage{titlesec}
\titleformat{\section}[block]
  {}{\S\thesection}{0.25cm}{\Large}
\title{Group Theory}
\author{Amit Rajaraman}
\date{December 2019}

\usepackage[utf8]{inputenc}
\usepackage{amsmath}
\usepackage{amssymb}
\usepackage{amsthm}
\usepackage{amsfonts}
\usepackage{enumerate}
\usepackage[margin=1in]{geometry}
\usepackage[colorlinks]{hyperref}
\usepackage{tikz}
\usepackage{titlesec}

\setlength\parindent{0pt}

\DeclareMathOperator{\Aut}{Aut}
\DeclareMathOperator{\Inn}{Inn}
\newcommand{\Mod}[1]{\ (\mathrm{mod}\ #1)}

\renewcommand{\qedsymbol}{$\blacksquare$}

\numberwithin{equation}{section}
\theoremstyle{definition}
\newtheorem{theorem}{Theorem}
\newtheorem{lemma}[theorem]{Lemma}
\newtheorem{corollary}[theorem]{Corollary}
\newtheorem{definition}{Definition}
\numberwithin{definition}{section}
\numberwithin{theorem}{section}
\newtheorem{exercise}{Exercise}
\newtheorem*{example}{Example}

\theoremstyle{remark}
\numberwithin{exercise}{section}
\newtheorem*{solution}{Solution}

\begin{document}
\maketitle
\tableofcontents
\clearpage

\input{introToGroups}
\section{Subgroups}

\subsection{Definitions and Basics}
Although we have defined subgroups in section $1$, we repeat the definition here.
\begin{definition}
Let $G$ be a group. A subset $H$ of $G$ is a subgroup of $G$ if $H$ is nonempty and it is closed under products and inverses. That is, $x,y\in H$ implies $x^{-1}\in H$ and $xy\in H$. If $H$ is a subgroup of $G$, we write $H\leq G$.
\end{definition}

If $H\leq G$ and $H\neq G$, we write $H<G$.

\begin{example}
    $\mathbb{Z}\leq\mathbb{Q}$ and $\mathbb{Q}\leq\mathbb{R}$ under the operation of addition.
    
    If $G=D_{2n}$, $H=\{1,r,r^2,\ldots,r^{n-1}\}$ is a subgroup of $G$.
\end{example}

Note that the relation $\leq$ is transitive. That is, if $K\leq H$ and $H\leq G$, then $K\leq G$.

\begin{theorem}[Subgroup Criterion]
A subset $H$ of a group $G$ is a subgroup if and only if
\begin{enumerate}
    \item $H\neq\emptyset$.
    \item for all $x,y\in H$, $xy^{-1}\in H$.
\end{enumerate}
Further, if $H$ is finite, then it suffices to check that $H$ is nonempty and is closed under multiplication.
\end{theorem}
\begin{proof}
    If $H\leq G$, the two given statements clearly hold as $H$ contains the identity of $G$ and is closed under inverses and multiplication.
    
    To prove the converse, let $x$ be any element of $H$ (which exists as $H\neq\emptyset$). We have $xx^{-1}\in H\implies 1\in H$. As $H$ contains $1$, for any element $h$ of $H$, $H$ contains $1h^{-1}=h^{-1}$, that is, it is closed under inverses. For any $x$ and $y$ in $H$, as $y^{-1}\in H$, we have that $x(y^{-1})^{-1}=xy\in H$, that is, $H$ is closed under multiplication.
    
    \vspace{1mm}
    To prove the second part, we see that $x,x^2,x^3,\ldots\in H$ for any $x\in H$. Using \ref{finGrpFinOrd}, we see that $x$ is of finite order $n$. Then $x^{-1}=x^{n-1}\in H$ so $H$ is closed under inverses.
\end{proof}

\begin{exercise}
    Let $G$ be a group and $H,K$ be subgroups of $G$. Show that $H\cup K$ is a subgroup if and only if $H\subseteq K$ or $K\subseteq H$.
\end{exercise}

\begin{exercise}
    Let $G$ be a group and $H,K$ be subgroups of $G$. Show that $H\cap K$ is also a subgroup of $G$.
\end{exercise}

\begin{exercise}
\label{subgrIntersubgr}
    Let $G$ be a group. Prove that the intersection of an arbitrary nonempty collection of subgroups of $G$ is again a subgroup of $G$.
\end{exercise}

\begin{exercise}
    Let $G$ be a group of order $n>2$. Show that $G$ cannot have a subgroup $H$ of order $n-1$.
\end{exercise}

\begin{exercise}
    Let $G$ be a group. Let $H=\{g\in G\mid |g|<\infty\}$. Show that $H\leq G$ if $G$ is abelian. In this case, $H$ is called the \textit{torsion subgroup} of $G$. Give an example where $G$ is non-abelian and $H$ is not a subgroup of $G$.
\end{exercise}

\begin{exercise}
    Let $H$ be a subgroup of $\mathbb{Q}$ under addition with the property that $\frac 1x\in H$ for every nonzero $x\in H$. Show that $H=\{0\}$ or $\mathbb{Q}$.
\end{exercise}

\subsection{Centralizers, Normalizers, Stabilizers and Kernels}

We now introduce some important subgroups.

\begin{definition}
    Let $G$ be a group and $A$ be any nonempty subset of $A$. Define
    $$C_G(A)=\{g\in G\mid gag^{-1}=a\text{ for all }a\in A\}.$$
    This subset is called the \textit{centralizer} of $A$ in $G$.
\end{definition}

Since $gag^{-1}=g$ if and only if $ga=ag$, $C_G(A)$ is the set of all elements that commute with every element of $A$.

Now observe that $C_G(A)$ is a subgroup of $G$ as first of all, $1\in C_G(A)$ so $C_G(A)\neq\emptyset$, and second of all, if $x,y\in C_G(A)$, we have $xax^{-1}=a$ and $yay^{-1}=a$, that is, $y^{-1}ay=a$ for all $a\in A$. We then have $a=xax^{-1}=x(y^{-1}ay)x^{-1}=(xy^{-1})a(xy^{-1})^{-1}$ so $xy^{-1}\in C_G(A)$. Thus, $C_G(A)\leq G$.

\begin{definition}
    Let $G$ be a group. Define
    $$Z(G)=\{g\in G\mid gx=xg\text{ for all }x\in G\}.$$
    This subset is called the \textit{center} of $G$.
\end{definition}

$Z(G)$ is the set of all elements that commute with every element of $G$.

As $Z(G)=C_G(G)$, we have $Z(G)\leq G$.

\begin{definition}
    Let $G$ be a group and $A$ be a subset of $G$. Define $gAg^{-1}=\{gag^{-1}\mid a\in A\}$. Define
    $$N_G(A)=\{g\in G\mid gAg^{-1}=A\}.$$
    This set is called the \textit{normalizer} of $A$ in $G$.
\end{definition}

The proof that $N_G(A)\leq G$ is similar to that we used to prove that $C_G(A)\leq G$.

Note that $C_G(A)\leq N_G(A)$.

\vspace{2mm}
If $G$ is an abelian group, $Z(G)=G$. Further, for any subset $A$ of $G$, $N_G(A)=C_G(A)=G$ as $gag^{-1}=gg^{-1}a=a$ for all $a\in A, g\in G$.

\begin{exercise}
    Show that the center of $D_8$ is $\{1,r^2\}$.
\end{exercise}

The fact that centralizers and normalizers are subgroups is in fact a special case of a results in group actions. 

We now introduce stablizers and kernels of group actions.

\begin{definition}
    Let $G$ be a group that acts on a set $S$. Let $s\in S$ be some fixed elements. Define
    $$G_s=\{g\in G\mid g\cdot s=s\}$$
\end{definition}

We shall now show that $G_s\leq G$. First of all, $1\in G_s$ by the definition of a group action. If $x,y\in G_s$, we have
\begin{align*}
    s &= 1\cdot s \\
      &= (x^{-1}x)\cdot s \\
      &= x^{-1}\cdot (x\cdot s) \\
      &= x^{-1}\cdot s
\end{align*}
so $x^{-1}\in G_s$ and
\begin{align*}
    (xy)\cdot s &= x\cdot(y\cdot s) \\
                &= x\cdot s \\
                &= s
\end{align*}
We see that $G_s$ is nonempty and is closed under inverses and multiplication. It is thus a subgroup of $G$.

Recall the definition of a \textit{kernel} of an action, \ref{defKerGrpAc}. Using \ref{subgrIntersubgr} and the fact that $G_s\leq G$ for all $s\in S$ yields the result that the kernel of any group action is a subgroup of the group.

\vspace{3mm}
We now see that $C_G(A)$ is merely the kernel of the group action of $G$ acting on $A$ as $g\cdot a= gag^{-1}$ (so it is a subgroup of $G$) and $N_G(A)$ is the stabilizer of the group action of $G$ acting on $\mathcal{P}(A)$ (the power set of $A$) as $g\cdot A=gAg^{-1}$ (so it is a subgroup of $G$).

\begin{exercise}
    Prove that $C_G(Z(G))=N_G(Z(G))=G$.
\end{exercise}

\begin{exercise}
    Prove that $H\leq N_G(H)$ for a subgroup $H$ of a group $G$.
\end{exercise}

\begin{exercise}
    For any subgroup $H$ of group $G$ and subset $A$ of $G$, define $N_H(A)=\{h\in H\mid hAh^{-1}=A\}$. Prove that $N_H(A)=N_G(A)\cap H$ and deduce that $N_H(A)\leq N_G(A)$.
\end{exercise}

\begin{exercise}
    Let $F$ be a field and the Heisenberg group $H(F)$ be defined as in \ref{defOfHeisenbergGrp}. Determine $Z(H(F))$ and prove that $Z(H(F))\cong (F,+)$.
\end{exercise}

\subsection{Cyclic Groups and Cyclic Subgroups}

\begin{definition}
    A group $H$ is \textit{cyclic} if there is some element $x\in H$ such that $H=\{x^n\mid n\in\mathbb{Z}\}$.
\end{definition}

In this case we write $H=\langle x\rangle$ and say that $H$ is \textit{generated} by $x$ and $x$ is a generator of $H$. The generator of a cyclic group need not be unique (as if $x$ is a generator, so is $-x$).

Note that any cyclic group is abelian.

\begin{example}
    The group $(\mathbb{Z},+)$ is generated by $1$ (here $1$ is the integer $1$ and not the identity).
\end{example}

\begin{theorem}
    Let $H=\langle x\rangle$. Then $|H|=|x|$ (where if one side of the inequality is infinite, so is the other).
\end{theorem}

\begin{proof}
    This proof is trivial and is left as an exercise to the reader.
\end{proof}

It is observed that there is a great deal of similarity between $H=\langle x\rangle$, where $|x|=n$, and $\mathbb{Z}/n\mathbb{Z}$. Both of them appear to have very similar structure. It turns out that these two groups are isomorphic, which we shall prove shortly. First, let us prove the following.

\begin{theorem}
    Let $G$ be an arbitrary group, $x\in G$, and let $m,n\in\mathbb{Z}$. If $x^n=1$ and $x^m=1$, then $x^d=1$, where $d=(m,n)$. In particular, if $x^m=1$ for some $m\in\mathbb{Z}$, then $|x|\mid m$.
\end{theorem}

\begin{proof}
    By the Euclidean algorithm, there exist integers $r$ and $s$ such that $d=mr+ns$. We have
    $$x^d=x^{mr+ns}=(x^m)^r(x^n)^s=1.$$
    This proves our first claim.
    
    Next, let $n=|x|$ and $x^m=1$. We have $x^d$=1, where $d=(|x|,m)$. Note that $0<d\leq |x|$ and $|x|$ is the smallest positive integer $k$ such that $x^k=1$. This implies that $d=|x|$ and $|x|=(|x|,m)$. Thus, $|x|\mid m$. 
\end{proof}

\begin{theorem}
    Any two cyclic groups of the same order are isomorphic. More specifically,
    \begin{enumerate}
        \item if $n\in\mathbb{Z}^+$ and $H=\langle x\rangle$ and $K=\langle y\rangle$ are both of order $n$, $H\cong K$.
        \item if $\langle x\rangle$ is an infinite cyclic group, $(\mathbb{Z},+)\cong\langle x\rangle$.
    \end{enumerate}
\end{theorem}

\clearpage
    
    % For $A$, let
    
    %      $\bar{A} = \{{a_1}^{\epsilon_1}{a_2}^{\epsilon_2}\cdots{a_n}^{\epsilon_n}$ $|$ $n\in\mathbb{Z}, n\geq0$ and $a_i\in A,\epsilon_i=\pm1$ for each $i\}$
    
    % where $\bar{A}={1}$ if $A=\emptyset$.
    
    % Then, $\bar{A} = \langle A\rangle$, where $\langle A\rangle$ represents the subgroup of $G$ generated by $A$ (the minimal subgroup of $G$ that contains $A$). Note that the $a_i$'s can be identical.
    
    % \textbf{\textit{Lagrange's Theorem}}:
    
    % If $G$ is a finite group and H is a subgroup of $G$, then $|H|$ divides $|G|$ and the number of left cosets of $H$ in $G$ equals $\frac{|G|}{|H|}$.
    % $\newline$
    % \textit{Proof outline}: The set of left cosets of $H$ in $G$ partition $G$. By definition of a left coset, the map $H\mapsto gH$ defined by $h\mapsto gh$ is a surjection from $H$ to the left coset $gH$. The left cancellation law implies this map is injective since $gh_1 = gh_2 \implies h_1 = h_2$. This proves that $H$ and $gH$ have the same order, $|gH|=|H|=n$.
    % Since $G$ is partitioned into $k$ disjoint subsets each of which has cardinality $n$, $|G|=kn$. Thus, $k = \frac{|G|}{|H|}$.
    
    % Note: The converse of Lagrange's Theorem (If $k\in\mathbb{Z}^{+}$ such that $k\mid|G|$, then there exists a subgroup of $G$ of order $k$) holds if $G$ is a finite abelian group.
    
    % To define a homomorphism from a group $G$ to $G'$, it is not enough to define the value of $\varphi$ at the generators of $G$, we must also ensure that the relations are satisfied. That is, if we have a relation $r=1$, where $r$ is some combination of generators, then we must also have that $\varphi (r)=1$.
    
    % Let $N$ be a subgroup of $G$. The following are equivalent:
    % \begin{enumerate}[i]
    %     \item $N \trianglelefteq G$ ($N$ is a normal subgroup of $G$)
    %     \item $N_G (N) = G$ ($N_G (N)$ is the normalizer in $G$ of $N$)
    %     \item $gN=Ng$ for all $g\in G$
    %     \item the operation of left cosets of $N$ in $G$ described by $uN\cdot vN = (uv)N$ (which is well-defined if and only if $gng^{-1}\in N$ for all $g\in G$ and all $n\in N$) makes the set of left cosets into a group.
    %     \item $gNg^{-1}\subseteq N$ for all $g\in G$ (this happens if and only if $gNg^{-1} = N$)
    %     \item $N$ is the kernel of some homomorphism.
    % \end{enumerate}
    
    % If $G = \langle S \rangle$, then if $N$ is a normal subgroup of G, $\frac GN = \langle \frac SN\rangle$.
    
    % $A\mathrel{\unlhd} B$ and $B\mathrel{\unlhd} C$ does \textit{not} imply that $A\mathrel{\unlhd} C$.
    % For example, $\langle s \rangle \mathrel{\unlhd} \langle s, r^2 \rangle \mathrel\unlhd D_8$ but $\langle s\rangle$ is not normal in $D_8$.
    
    % In abelian groups, every subgroup is normal.
    
    % If $\frac{G}{Z(G)}$ is cyclic, $G$ is abelian.
    
    % \textit{Cauchy's Theorem}: If $G$ is a finite group and $p$ is a prime dividing $|G|$, then $G$ has an element of order $p$. (Page 96 Q9, Dummit and Foote)
    
    % If $H$ and $K$ are subgroups of a group, $HK$ is a subgroup if and only if $HK=KH$.
    
    % If $H$ and $K$ are subgroups of $G$ and $H\leq N_G(K)$, then $HK$ is a subgroup of $G$. In particular, if $K\mathrel{\unlhd} G$ then $HK\leq G$ for any $H\leq G$.
    
    % Let $H\leq G$. The set of left cosets of $H$ in $G$ is in bijection with the set of right cosets of $H$ in $G$ ($x\mapsto x^{-1}$ maps each left coset to a right coset).
    
    % \begin{enumerate}
    
    % \item\textit{The First Isomorphism Theorem}: If $\varphi: G\to H$ is a homomorphism of groups, then $\ker \varphi \unlhd G$ and $G/\ker \varphi \cong \varphi(G).$
    
    % Corollary: $|G:\ker\varphi|=|\varphi(G)|$. $\varphi$ is injective if and only if $\ker \varphi = 1$.
    
    % \item\textit{The Second or Diamond Isomorphism Theorem}: Let $G$ be a group, let $H$ and $K$ be subgroups of $G$ and assume $H\leq N_{G}(K)$. Then $HK\leq G$, $K\unlhd HK$, $H\cap K\unlhd H$ and $HK/K\cong H/H\cap K$. This theorem's name can be understood from Fig. 1.
    
    % \begin{figure}[ht!]
    %     \centering
    %     \includegraphics{img/diamondIsoThm.pdf}
    %     \caption{Diamond Isomorphism Theorem}
    %     \label{Fig. 1}
    % \end{figure}
    
    % \item\textit{The Third Isomorphism Theorem}: Let $G$ be a group and $H$ and $K$ be normal subgroups of $G$ with $H\leq K$. Then $K/H\unlhd G/H$ and $(G/H)/(K/H)\cong G/K$.
    
    % \textit{The Fourth or Lattice Isomorphism Theorem}: Let $G$ be a group and let $N$ be a normal subgroup of $G$. Then there is a bijection from the set of subgroups $A$ of $G$ which contain $N$ onto the set of subgroups $\overline{A}=A/N$ of $G/N$. In particular, every subgroup of $\overline{G}$ is of the form $A/N$ for some subgroup $A$ of $G$ containing $N$ (namely, its preimage in $G$ under the natural projection homomorphism from $G$ to $G/N$). This bijection has the following properties: for all $A,B\leq G$ with $N\leq A$ and $N\leq B$,
    % \begin{enumerate}[i]
    %     \item $A\leq B$ if and only if $\overline{A}\leq\overline{B}$,
    %     \item if $A\leq B$, then $|B\mathrel{:}A|=|\overline{B}\mathrel{:}\overline{A}|$,
    %     \item $\overline{\langle A, B\rangle} = \langle\overline{A}, \overline{B}\rangle$,
    %     \item $\overline{A\cap B}=\overline{A}\cap\overline{B}$
    %     \item $A\unlhd G$ if and only if $\overline{A}\unlhd \overline{G}$
    % \end{enumerate}
    % \end{enumerate}
    
    % If $H$ is a normal subgroup of $G$ of prime index $p$ then for all $K\leq G$, either
    % \begin{enumerate}[i]
    %     \item $K\leq H$ or
    %     \item $G=HK$ and $|K:H\cap K|=p$.
    % \end{enumerate}
    
    % In a group $G$, a sequence of subgroups $$1=N_{0} \leq N_{1} \leq N_{2} \leq\cdots\leq N_{k-1}\leq N_{k}=G$$ is called a \textit{composition series} if $N_{i} \unlhd N_{i+1}$ and $N_{i+1}/N_{i}$ is a simple group, $0\leq i\leq k-1$. If the above sequence is a composition series, the quotient groups $N_{i+1}/N_{i}$ are called \textit{composition factors} of $G$.
    
    % \textit{Jordan-H{\"o}lder Theorem:} Let $G$ be a finite group with $G\neq 1$. Then
    % \begin{enumerate}[i]
    %     \item $G$ has a composition series.
    %     \item The composition factors in a composition series are unique. Namely, if $1=N_{0} \leq N_{1} \leq N_{2} \leq\cdots\leq N_{r-1}\leq N_{r}=G$ and $1=M_{0} \leq M_{1} \leq M_{2} \leq\cdots\leq M_{s-1}\leq M_{s}=G$, then $r=s$ and there is some permutation $\pi$ of $\{1,2,\cdots, r\}$ such that $M_{\pi(i)}/M_{\pi(i)-1}\cong N_{i}/N_{i-1}$. Note that the series itself need not be unique, but the composition factors are unique.
    % \end{enumerate}    
    
    % \textit{Feit-Thompson}: If $G$ is a simple group of odd order, then $G\cong Z_p$ for some prime $p$.
    
    % A group $G$ is \textit{solvable} if there is a composition series of $G$ such that every composition factor of $G$ is abelian.
    
    % Let $G$ be a finite group. The following are equivalent:
    % \begin{enumerate}[i]
    %     \item $G$ is solvable.
    %     \item $G$ has a composition series such that every composition factor is cyclic.
    %     \item All composition factors of $G$ are of prime order.
    %     \item $G$ has a chain of subgroups: $1=N_{0} \leq N_{1} \leq N_{2} \leq\cdots\leq N_{t-1}\leq N_{t}=G$ such that each $N_i$ is a normal subgroup of $G$ and $N_{i+1}/N_{i}$ is abelian, $0\leq i\leq t-1$.
    %     \item For every divisor $n$ of $|G|$ such that $\left(n,\frac{|G|}{n}\right)=1$, $G$ has a subgroup of order $n$.
    %     \item There exists a normal subgroup $N$ of $G$ such that both $N$ and $G/N$ are solvable.
    % \end{enumerate}
    
    % The permutation $\sigma$ is odd if and only if the number of cycles of even length in its cycle decomposition is odd.
    
    % $A_n$, the alternating group of degree $n$, is a non-abelian simple group for all $n\geq5$.
    
    % An action of $G$ on $A$ may also be viewed as a faithful action of $G/\ker \varphi$ on $A$.

    % Let $G$ be a group acting on a nonempty set $A$. For each $g\in G$, the map $\sigma_{g}:A\to A$ defined by $\sigma_{g}(a)=g\cdot a$ is a permutation of $A$. There is a homomorphism associated with this action of $G$ on $A$ given as $\varphi:G\to S_A$ defined by $\varphi(g)=\sigma_{g}$ called the \textit{permutation representation} associated with this action. The kernel of this action is the same as the kernel of $\varphi$.
    
    % \begin{definition}
    % If $G$ is a group, a \textit{permutation representation} of $G$ is any homomorphism of $G$ into the symmetric group $S_A$ for some nonempty set $A$. We shall say that the given action \textit{affords} or \textit{induces} the associated permutation representation of $G$.
    % \end{definition}
    
    % Let $G$ be a group acting on the nonempty set $A$. The relation on $A$ defined by $a\sim b$ if and only if $a=g\cdot b$ for some $g\in G$ is an equivalence relation. For each $a\in A$, the number of elements in the equivalence class containing $a$ is $|G:G_a|$, where $G_a$ is the stabilizer of $a$.
    
    % Let $G$ be a group, let $H$ be a subgroup of $G$ and let $G$ act by left multiplication on the set $A$ of left cosets of $H$ in $G$. Let $\pi_H$ be the associated permutation representation afforded by this action. Then
    % \begin{enumerate}[i]
    %     \item $G$ acts transitively on $A$.
    %     \item the stabilizer in $G$ of the point $1H\in A$ is the subgroup $H$.
    %     \item the kernel of the action (i.e., the kernel of $\pi_H$) is $\cap_{x\in G}xHx^{-1}$ and $\ker\pi_H$ is the largest normal subgroup of $G$ contained in $H$.
    % \end{enumerate}
    
    % \textit{Cayley's Theorem}: If $G$ is a group of order $n$, then $G$ is isomorphic to a subgroup of $S_n$.
    
    % \textit{Proof}: Just put $H=\{1\}$ in the previous point to get a homomorphism from $G$ to $S_G$. Since the kernel is contained in $H=\{1\}$, $G$ is isomorphic to its image in $S_G$.
    
    % If $G$ is a finite group and $p$ is the smallest prime dividing $|G|$,  any subgroup of index $p$ is normal. Note that, however, a group need not necessarily have a subgroup of index $p$.
    
    % \begin{proof}
    % We have $H\leq G$ and $|G:H|=p$. Let $\pi_H$ be the permutation representation given by multiplication on the set of left cosets of $H$ in $G$. Let $K=\ker \pi_H$ and $|H:K|=k$. Then $|G:K|=|G:H||H:K|$ As there are $p$ left cosets of $H$ in $G$, we have that $G/K$ is isomorphic to a subgroup of $S_p$ (the image of $G$ under $\pi_H$). This implies $pk\mid p!$ and $k\mid (p-1)!$. The minimality of $p$ implies that $|H:K|=1$ and $H=K\unlhd G$.
    % \end{proof}
    
    % Two subsets $S$ and $T$ of $G$ are said to be conjugate in $G$ if there exists $g\in G$ such that $T=gSg^{-1}$.
    
    % The number of conjugates of a subset $S$ of $G$ is the index of the normalizer of $S$, $|G:N_G(S)|$. It follows that the number of conjugates of an element $s$ of $G$ is the index of the centralizer of $s$, $|G:C_G(s)|$. (as $N_G(\{s\})=C_G(s)$)
    
    % \textit{The Class Equation}: Let $G$ be a finite group and let $g_1, g_2, \cdots g_r$ be representatives of the distinct conjugacy classes of $G$ not contained in the center $Z(G)$ of G. Then $$|G|=|Z(G)|+\sum_{i=1}^{r}|G:C_G(g_i)|$$ Note that this is useless for abelian groups.
    
    % If $p$ is a prime and $P$ is a group of prime power order $p^{\alpha}$ for some integer $\alpha\geq 1$, then $P$ has a nontrivial center: $Z(P)\neq \{1\}$.
    
    % \begin{corollary}
    % If $|P|=p^2$ for some prime $p$, then $P$ is abelian. More precisely, $P$ is isomorphic to either $Z_{p^2}$ or $Z_p\times Z_p$.
    % \end{corollary}
    
    % Let $\tau, \sigma$ be members of the symmetric group $S_n$. Then, $\tau\sigma\tau^{-1}$ is obtained from $\sigma$ by replacing each entry $i$ in the cycle decomposition of $\sigma$ with $\tau(i)$.
    
    % If $\sigma\in S_n$ is the product of disjoint cycles of lengths $n_1,n_2,\cdots, n_r$ with $n_1\leq n_2\leq\cdots\leq n_r$ (including its $1$-cycles) then the integers $n_1,n_2,\cdots,n_r$ are called the cycle type of $\sigma$.
    
    % Two elements of $S_n$ are conjugate if and only if they have the same cycle type. The number of conjugacy classes of $S_n$ equals the number of partitions of $n$.
    
    % If $\sigma$ is an $m$-cycle in $S_n$, then $C_{S_n}(\sigma)=\{\sigma^i\tau\mid 0\leq i\leq m-1, \tau\in S_{n-m}\}$ where $S_{n-m}$ denotes the subgroup of $S_n$ which fixes the integers appearing in the $m$-cycle $\sigma$. $|C_{S_{n}}(\sigma)|=m\cdot(n-m)!$.
    
    % If $H\unlhd G$, then for every conjugacy class $\mathcal{K}$ of $G$, either $\mathcal{K}\subseteq H$ or $\mathcal{K}\cap H = \emptyset$.
    
    % If $Z(G)$ is of index $n$, any conjugacy class of $G$ is of order atmost $n$. 
    
    % Assume $H\unlhd G$, $\mathcal{K}$ is a conjugacy class of $G$ contained in $H$ and $x\in\mathcal{K}$. Then, $\mathcal{K}$ is a union of $k$ conjugacy classes of equal size in $H$, where $k = |G\mathrel{:}HC_G(x)|$.
    
    % Let $H\unlhd G$. Then $G$ acts by conjugation on $H$ as automorphisms of $H$. More specifically, the action of $G$ on $H$ by conjugation is defined for each $g\in G$ by $h\mapsto ghg^{-1}$ for each $h\in H$. For each $g\in G$, conjugation by $g$ is an automorphism of $H$. The permutation representation afforded by this action is a homomorphism of $G$ into $\operatorname{Aut}(H)$ with kernel $C_G(H)$. In particular, $G/C_G(H)$ is isomorphic to a subgroup of $\Aut(H)$.
    
    % \begin{corollary}
    % For any $H\leq G$, $N_G(H)/C_G(H)$ is isomorphic to a subgroup of $\Aut(H)$. In particular, putting $H=G$, $G/Z(G)$ is isomorphic to a subgroup of $\Aut(G)$.
    % \end{corollary}
    
    % Let $G$ be a group and $g\in G$. Conjugation by $g$ is called an \textit{inner automorphism} of $G$ and the subgroup of $\Aut(G)$ consisting of all inner automorphisms of $G$ is called $\Inn(G)$. We have that $\Inn(G)\cong G/Z(G)$ and $\Inn(G)\unlhd\Aut(G)$ ($\Aut(G)/\Inn(G)$ is called the outer isomorphism group of $G$)
    
    % $\Aut(Z_n)\cong (\mathbb{Z}/n\mathbb{Z})^{\times}$
    
    % \begin{definition}
    % A subgroup $H$ of $G$ is called \textit{characteristic} in $G$, denoted by $H \operatorname{char} G$, if every automorphism of $G$ maps $H$ to itself, i.e., $\sigma(H)=H$ for all $\sigma\in\Aut(G)$.
    % \end{definition}
    
    % Then,
    % \begin{enumerate}[i]
    %     \item characteristic subgroups are normal,
    %     \item if $H$ is the unique subgroup of $G$ of a given order, then $H \operatorname{char} G$,
    %     \item if $K\operatorname{char} H$ and $H\unlhd G$, then $K\unlhd G$ and
    %     \item if $K\operatorname{char} H$ and $H\operatorname{char} G$, then $K\operatorname{char} G$.
    % \end{enumerate}
    
    % Let $G$ be a group of order $pq$, where $p$ and $q$ are primes (not necessarily distinct) with $p\leq q$. If $p\nmid q-1$, $G$ is cyclic. The proof that $G$ is abelian is as follows.
    
    % \begin{proof}
    % If $Z(G)\neq 1$, then Lagrange's Theorem forces $G/Z(G)$ to be cyclic and hence $G$ to be abelian. Hence we may assume $Z(G)=1$.
    
    % If every nonidentity element of $G$ has order $p$, then the centralizer of every nonidentity element has index $q$, so the class equation for $G$ reads $pq = 1 + kq$. This is impossible since $q$ divides $pq$ and $kq$ but not $1$. Thus $G$ contains an element $x$ of order $q$.
    
    % Let $H=\langle x\rangle$. Since $H$ has index $p$ and $p$ is the smallest prime that divides $|G|$, $H$ is normal in $G$. Since $Z(G)=1$, we must have $C_G(H)=H$. Thus $G/H=N_G(H)/C_G(H)$ is a group of order $p$ isomorphic to a subgroup of $\Aut(H)$. But $\Aut(H)$ has order $\varphi(q)=q-1$ which by Lagrange's Theorem would imply $p\mid q-1$, contrary to the assumption.
    % \end{proof}
    
    % It can further be checked that every such group is cyclic.
    
    % Descriptions of isomorphism types of some automorphism groups:
    
    % \begin{itemize}
    %     \item The automorphism group of the cyclic group of order $p^n$ is cyclic of order $p^{n-1}(p-1)$.
    %     \item For all $n\geq 3$ the automorphism type of the cyclic group of order $2^n$ is isomorphic to $Z_{2}\times Z_{2^{n-2}}$, and in particular is not cyclic but has a cyclic subgroup of index $2$.
    %     \item Let $p$ be a prime and let $V$ be an abelian group (written additively) with the property that $pv=0$ for all $v\in V$. If $|V|=p^n$, then $V$ is an n-dimensional vector space over the field $\mathbb{F}_p=\mathbb{Z}/p\mathbb{Z}$. The automorphisms of $V$ are precisely the nonsingular linear transformations from $V$ to itself, that is, $$\Aut(V)\cong GL(V)\cong GL_n(\mathbb{F}_p).$$ In particular, the order of $\Aut(V)$ is $(p^{n}-1)(p^{n}-p)(p^{n}-p^{2})\cdots(p^n-p^{n-1})$.
    %     \item For all $n\neq 6$ we have $\Aut(S_{n})=\Inn(S_{n})\cong S_{n}$. For $n=6$, we have $|\Aut(S_{6})\mathrel{:}\Inn(S_{6})|=2$.
    %     \item $\Aut(D_8)\cong D_8$ and $\Aut(Q_8)\cong S_4$.
    % \end{itemize}
    
    % \begin{definition}
    % Let $G$ be a group and $p$ be a prime.
    % \begin{itemize}
    %     \item A group of order $p^{\alpha}$ for some $\alpha\geq 1$ is called a \textit{$p$-group}.
    %     \item If $G$ is a group of order $p^{\alpha}m$, where $p\nmid m$, then a subgroup of order $p^{\alpha}$ is called a \textit{Sylow $p$-subgroup} of $G$.
    %     \item The set of Sylow $p$-subgroups will be denote by $Syl_{p}(G)$ and the number of Sylow $p$-subgroups of $G$ will be denoted by $n_{p}(G)$ (or just $n_p$).
    % \end{itemize}
    % \end{definition}
    
    % \textit{Sylow's Theorem}: Let $G$ be a group of order $p^{\alpha}m$ where $p$ is a prime that does not divide $m$.
    % \begin{enumerate}
    %     \item Sylow $p$-subgroups of $G$ exist, i.e., $Syl_p(G)\neq\emptyset$.
    %     \item If $P$ is a Sylow $p$-subgroup of $G$ and $Q$ is any $p$-subgroup of $G$, then there exists $g\in G$ such that $Q\leq gPg^{-1}$, i.e., $Q$ is contained in some conjugate of $P$. In particular, any two Sylow $p$-subgroups of $G$ are conjugate in $G$.
    %     \item The number of Sylow $p$-subgroups of $G$ is of the form $1+kp$, i.e., $$n_p\equiv 1\Mod p.$$ Further, $n_p$ is the index in $G$ of the normalizer $N_G(P)$ for any Sylow $p$-subgroup $P$, hence $$n_p\mid m.$$
    % \end{enumerate}
    
    % Any two Sylow $p$-subgroups of a group (for the same prime $p$) are isomorphic.
    
    % Let $P\in Syl_p(G)$. If $Q$ is any $p$-subgroup of $G$, then $Q\cap N_G(P)=Q\cap P$.
    
    % Let $P$ be a Sylow $p$-subgroup of $G$. Then the following are equivalent:
    % \begin{enumerate}
    %     \item $P$ is the unique Sylow $p$-subgroup of $G$, i.e., $n_p=1$
    %     \item $P\unlhd G$
    %     \item $P\operatorname{char} G$
    %     \item All subgroups generated by elements of $p$-power order are $p$-groups, i.e., if $X$ is any subset of $G$ such that $|x|$ is a power of $p$ for all $x\in X$, then $\langle X\rangle$ is a $p$-group.
    % \end{enumerate}
    
\end{document}
